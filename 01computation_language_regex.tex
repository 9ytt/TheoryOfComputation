%#!platex main.tex

\chapter{オートマトンと言語理論}

\section{計算とは}

\subsection{計算の仕様}
計算とは,以下のような仕様を持つものである.
\[
 仕様S: \N \rightarrow \N
\]
つまり,「ある自然数をある自然数に対応させるのが計算」と言える.

この講義の文脈では,「ある自然数」を0/1の有限列で表す(何進数かは本質的で
はない).

この仕様の性質から,以下のことが言える.
\begin{itemize}
 \item 仕様も所詮自然数から自然数の対応なので,仕様自体も自然数で表現で
       きる.\\
       ex: 「000 $\rightarrow$ 011」という仕様を「1100」と表す(インデッ
       クスをつける)
 \item あらゆる仕様の集合は連続体濃度. \footnote{よく分からないけど$\R$の
       濃度らしい \cite{wikipedia}}
 \item プログラムは仕様に含まれるから,プログラムは加算集合.
       \footnote{プログラムで記述できるのは,仕様の集合のごく一部}
\end{itemize}

\subsection{計算の能力} \label{sec:計算の能力}
実は出力がNビットの計算も,出力1ビットの計算に帰着させることができる.例
えば,
\[
 計算A: 入力010, 出力110
\]
は,
\begin{eqnarray*}
 計算B_1: 入力010, 出力1\\
 計算B_2: 入力010, 出力1\\
 計算B_3: 入力010, 出力0
\end{eqnarray*}
を逐次的に実行する計算Bに置き換えることができる.

こうして,出力を1ビットにできるよというと,この先の講義でよく出てくるオー
トマトンの話がちょっとありがたく思える.

出力が1ビットということは,出力は真偽値をとると考えられる.

オートマトンは,「入力を受理するかどうか」を判定する.この真偽の判定を,0を出力す
るか1を出力するかに対応させると,オートマトンは(ある程度)どのような計算でもできる
ということがわかる.

\section{言語 (Language)}
\mystrong{言語}とは,有限種の記号列の有限列の集合と定義できるが,その定義を理解する
ために必要な用語をまず下に記す.

\begin{description}
 \item [$\Sigma$] \mbox{} \\
 用いる記号の有限集合.\mystrong{アルファベット}
       \footnote{別に$abc\cdots zABC\cdots Z$以外でも,記号ならなんでもOK.「欝」を
       記号に含むアルファベットも構成可}ともいう.
             \begin{myexample}{英字と数字を記号とするアルファベット}
              \[
              \Sigma = \{a,b,c,\cdots ,z,A,B,C,\cdots ,Z,0,1,\cdots ,9\}
              \]
             \end{myexample}

 \item [\mystrong{記号列(string)}] \mbox{} \\
 記号の有限の並び. \footnote{言語論で
             「並び」というと,順序に意味のある列という意味.「集合」
             は,順序に意味のない列.}
             \begin{myexample}{上の例での$\Sigma$上の記号列}
              \[
              \varepsilon, a, b, c, \cdots , 9, aa, ab, \cdots  ,afbafe132, \cdots 
              \]
              空列$\epsilon$(下記)を含むことに注意.
             \end{myexample}

 \item[長さ(length)] \mbox{} \\
            記号列中の記号の数.$|w|$が$w$の長さを表す.
            \begin{myexample}{ある記号列の長さ}
             \[
              |afbafe132| = 9
             \]
            \end{myexample}

 \item[空列(empty string)] \mbox{} \\
長さ0の記号列.\mystrong{$\varepsilon$}で表す.
\end{description}

ここにきて,\mystrong{言語}の定義,「有限種の記号列の有限列の集合」を見
直すと,何となく分かるのではないかと思う.
\begin{myexample}{ある記号列集合で定義される言語}
 以下のような,英字2文字で成り立つような記号列の集合を考える.
 \[
  aa, ab, ac, \cdots , ba, bb, bc, \cdots , zx, zz
 \]
 この記号列の集合によって定義される言語には,例えば以下のようなものがあ
 る.
 \[
  \Sigma_1 = \{aa\}
 \]
 \[
  \Sigma_2 = \{ex, gy\}
 \]
 \[
  \Sigma_3 = \{an, ax, if, up\}
 \]
\end{myexample}

ここで,今後の話でよく出てくる,$\Sigma^*$という言語を見てみる.
\begin{description}
 \item [\mystrong{$\Sigma^*$}] \mbox{} \\
 $\Sigma$上の記号列全てを含む集合.これは,
            言語になっている.
            \begin{myexample}{アルファベット$\{a\}$で定義される言語}
             いま,単純なアルファベット
             \[
              \Sigma = \{a\}
             \]
             を考える.このアルファベット上の記号列は,
             \[
              \varepsilon, a, aa, \cdots 
             \]
             であるので,$\Sigma$に対しては
             \[
              \Sigma^* = \{\varepsilon, a, aa, \cdots \}
             \]
             が定義される.
            \end{myexample}
\end{description}

\subsection{言語で定義される言語}
\begin{description}
 \item [\mystrong{連接(concatenation)}] \mbox{} \\
 2つの言語$L_1$と$L_2$の連接は,
            \[
             L_1 L_2 = \{vw | v \in L_1 , w \in L_2\}
            \]
            で定義される.
            \begin{myexample}{単純な連接}
             \[
              abc \in L_1 , \hspace{2ex} defg \in L_2
             \]
             であれば,
             \[
              abcdefg \in L_1 L_2
             \]
             である.ただし,$L_1 L_2$は$abc$や$defg$を含まないことに注
             意.
            \end{myexample}
            
 \item [\mystrong{閉包(closure)}] \mbox{} \\
 $\Sigma$上の言語$L (\subset \Sigma^*) $の
            要素をいくつでも並べたもの.
            \[
             L^0 = \{\epsilon\}, \hspace{2ex} L^1 = \{w | w \in L\}, \hspace{2ex} L^2 = \{vw
            | v, w \in L\}, \hspace{2ex} \cdots 
            \]
            とするとき,閉包は
            \[
            \displaystyle
             L^* = L^0 \cup L^1 \cup L^2 \cup \cdots = \cup^{\infty}_{i
            = 0} L^i
            \]
            と定義される.
            \begin{myexample}{単純な閉包}
             \[
             L = \{abc, efg, hij\}
             \]
             であったとき,
             \[
              L^* = \{\epsilon, abc, abcabc, \cdots, abcefg, \cdots,
             hijabcefg, \cdots\}
             \]
             などである.
            \end{myexample}
            
 \item [\mystrong{正閉包(positive closure)}] \mbox{} \\
            \[
            \displaystyle
             L^+ = \cup^{\infty}_{i = 1} L^i
            \]
            で定義される.
\end{description}

\section{正規表現(Regular expression, RE)}
$\Sigma$上の正規表現\footnote{言語論の業界ではREと
 略すようだが,コンピュータの業界ではregexとかregexpとか略すことも多い.}
 とは,以下のいずれかである.
\begin{enumerate}
 \item $\phi$ : 表す言語(=記号列の集合)は空集合.
 \item 空列$\varepsilon$ : 表す言語は$\{\varepsilon\}$
 \item $\Sigma$の各要素$a$に対して$a$ : 表す言語は{a}
 \item $r, s$がそれぞれ言語$R, S$を表す正規表現であるとき,
       \begin{itemize}
        \item $(r+s)$ : $R\cup S$を表す.
        \item $(rs)$ : $RS$を表す.
        \item $(r*)$ : $R^{*}$を表す.
       \end{itemize}
\end{enumerate}
