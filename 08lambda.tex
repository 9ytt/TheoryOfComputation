\chapter{$\lambda$計算(Lambda calculus)}
\mystrong{$\lambda$計算(lambda calculus)}\footnote{$\lambda$計算を実装したプログラミング言語としてはLispがとても有
名です.この授業の勉強で使えとは当然言いません
が,\cite{Gerald_Julie_Harold_和田200002}を読んでみると(プログラム的に)
見える世界が変わってきます.個人的イチオシ.}とは,帰納的関数を数と無関係に
定義する枠組みである.

\section{$\lambda$抽象化(Lambda abstraction)}
計算の手続きに名前をつけることができる.例えば,
\[
 a \rightarrow 1+1
\]
など.ここで更に
\[
 b \rightarrow 1+1
\]
とすると,これらの手続きは名前は違えど,表す手続きの内容は同じである.こ
のような手続きを一緒くたにまとめることを,\mystrong{$\lambda$抽象化(Lambda abstraction)}という.
\footnote{$\lambda$抽象化については授業であまり説明がなく,この辺は自分の理
解で書きました.もしかしたら間違ってるかもしれないので注意.}
\footnote{$\lambda$抽象化を利用すれば,いちいち名前をつけずに手続きを定義で
きます.これはプログラムでは無名関数という形で出てき
て,Lisp,Javascript,その他最近の言語では無名関数が使えます.無名関数の
概念は,次のような感じ($do\_anything\_to\_x$を呼び出す際の第1引数が無名関数
になっています.)
\begin{equation} \label{eq:08koukai}
 do\_anything\_to\_x (\{ x * x \}, x); 
\end{equation}
}

\section{$\lambda$式(Lambda expression)}
\mystrong{$\lambda$式(lambda expression)}は,
\[
\lambda x. 定義
\]
と表記する.
 \footnote{$\lambda x. 定義$の$x$は変数名}
$\lambda x.定義$は,
\[
 「ひとつの引数を取って定義の内容を返す関数」を表す式
\]

\begin{myexample}{$\lambda$式の例}
\[
 f(x) = 3x + 2 \hspace{2ex} g(x) = 3x + 2\hspace{2ex}\cdots 
\]
 などの,「1変数をとり,その変数を3倍した数に2を足した数を返す\mystrong{関数}」は,全て
 \[
  \lambda x. (3x+2)
 \]
 という\mystrong{式}で表せる.
\end{myexample}

\section{高階関数(Higher-order function)}
\mystrong{高階関数(Higher-order function)}とか,関数を引数や結果とする関
数である.\footnote{脚注に書いた$do\_anything\_to\_x$ (式\ref{eq:08koukai})も一種の高階
関数です.}

\begin{myexample}{ある関数を2回適用する関数を高階関数として記述する}
 \[
  g(x) = f(f(x))
 \]
 この$g$と同じ意味を持つ,
 \[
  twice(f)
 \]
 という関数を作りたい.これを$\lambda$式として書くと,
 \[
  twice = \lambda f . (\lambda x . fx)
 \]
 となる.\footnotemark
\end{myexample}
\footnotetext{$\lambda$式の表記法は勉強しておきましょ
う.\mystrong{$\lambda$式で解答するような問題が試験にも出題され
る模様です.}}

\section{カリー化(Currying)}
\mystrong{カリー化(Currying)}とは,複数引数の関数を単一引数で表現するこ
とをいう.

\begin{myexample}{簡単なカリー化}
 \[
  f(x, y, z) = x + y \times z
 \]
 という3変数関数を,
 \[
  f_{x,y} (z) = x + y \times z
 \]
 という風に1引数関数にすることもできる($x,y$を固定した).これがカリー化.

 この$f_{x,y}$は,$\lambda$式では
 \[
  \lambda x . (\lambda y . (\lambda z . (x + y \times z)))
 \]
 と表記できる.
\end{myexample}

\section{$\lambda$式について覚えておきたいこと}
\subsection{$\lambda$式の略記法}
$\lambda$式の基本形は$\lambda 変数名 . 定義$であるが,「概念的に複数引数
を取る$\lambda$式」は,次のように表記できる.
\[
\lambda x_1 x_2 \cdots x_n . 定義 \equiv \lambda x_1 . (\lambda
x_2  (\cdots . (\lambda x_n . 定義))) \footnote{カリー化の逆適用ですね.}
\]

\begin{myexample}{略記法の簡単な例}
\[
 \lambda xyz . (x+y \times z) \equiv  \lambda x . (\lambda y . (\lambda z . (x + y \times z))) 
\]
\end{myexample}


\subsection{関数の適用順}
関数を連続して適用するとき,その適用順は次のとおりである.
\[
 M_1 M_2 M_3 \cdots M_n \equiv (((\cdots (M_1M_2)M_3)\cdots)M_n)
\]
ただし,この適用順を明示的に変更したいときは,括弧を使う.

\begin{myexample}{関数の適用順の例}
\[
 fffx \equiv (((ff)f)x) \footnote{「関数に対して関数を入力しているので変に思
 うかもしれませんが,それが高階関数というものです.」}
\]
 高校で習った「合成関数」を再現したいときは,
 \[
 f(f(fx))
 \]
 という風に括弧でくくります.
\end{myexample}

\subsection{基本的な関数}
\[
 I \equiv \lambda x . x\hspace{2ex};Identity
\]
\[
 K \equiv \lambda xy . x\hspace{2ex};定数値を返す関数 
\]

\subsection{引数への適用} \footnote{勝手に付け加えました.}
$\lambda$式の引数に値を適用するときは,
\[
 (\lambda 引数 . 定義) 適用する値
\]
と書く.
\begin{myexample}{引数への値適用の例}
 \begin{eqnarray*}
  (\lambda x . x + 3) 5 &=& 5 + 3 \\
  &=& 8
 \end{eqnarray*}
\end{myexample}

\section{$\beta$変換($\beta$ reduction)}
式$M$の中の全ての自由な \footnote{$\lambda$式の引数になっている引数
は,$\lambda$式によって束縛されています.つまり,定義の部分の変数を自由
な変数と言います.ところで,引数は$\lambda$式に束縛されることを利用し
て,$\lambda$式によってロー
カル変数が作れます.Lispのletですね.胸熱.}$x$を式$N$で置き換えた結果を,
\[
 M[x:=N]
\]
と書く.

\begin{myexample}{略記法の例}
\[
 \lambda x . (x + (\lambda x . x) 5)
\]
について,
\[
  M = x + (\lambda x . x) 5
\]
 とし,
\[
 \lambda x . (x + (\lambda x . x) 5) = \lambda x . M
\]
とすると,
 \[
 M[x:=3] = 3 + (\lambda x . x)5 = 3 + 5 = 8
 \]
\end{myexample}


この略記法を用いて,以下のように\mystrong{$\beta$変換($\beta$
reduction)}が定義される.\footnote{$\alpha$変換もある.それは,
\[
 \lambda x. x \underset{\alpha}{\longrightarrow} \lambda y . y
\]
のように定義される.}

\begin{description}
 \item[引数渡し]  \mbox{} \\
            $(\lambda x . M) N \underset{\beta}{\longrightarrow} M[x:=N]$ 

 \item[$M \underset{\beta}{\longrightarrow} N$ならば,] \mbox{} \\
            \begin{description}
             \item[定義の中での変換] \mbox{} \\
                        $\lambda x . M \underset{\beta}{\longrightarrow} \lambda x . N$ 

             \item[引数の変換]  \mbox{} \\
                        $ PM \underset{\beta}{\longrightarrow} PN$ 
             \item[関数の変換]  \mbox{} \\
                        $MP \underset{\beta}{\longrightarrow} NP$
            \end{description}
\end{description}


\begin{itemize}
 \item $\beta$変換の対象となる式を$\beta$基($\beta$-redex)という.
 \item 式$P$から有限回の$\beta$変換で,
       \[
        P = P_1 \underset{\beta}{\longrightarrow} P_2 \underset{\beta}{\longrightarrow} \cdots
       \underset{\beta}{\longrightarrow} P_n = Q
       \]
       となるとき,この列$P_1 \cdots P_n$を$\beta$変換列といい,
       \[
        P \underset{\beta}{\longrightarrow} Q
       \]
       と表す.
\end{itemize}

\section{Church-Rosserの定理}
\mystrong{Church-Rosserの定理}とは,次のものである.

\begin{mytheorem}
同じ式から$\beta$変換列で導かれた二つの式からは,必ず$\beta$変換列によっ
て同じ式を導くことができる.
\end{mytheorem}
なお,この性質を合流性(Confluence)と呼ぶこともある.

図
\ref{fig:img/08beta_reduction.eps}は,合流性を表すものである.図中の矢印
は全て$\beta$変換を表す.例えば,2つの黄色い式から,赤い式に「合流」して
いることが見て取れる.

\myfigure{img/08beta_reduction.eps}{$\beta$変換列の合流性}


\section{正規形(Normal form)}
$\beta$基を含まない$\lambda$式,すなわち,それ以上$\beta$変換できない式を\mystrong{正規形(normal form)}であるとい
う.

\begin{myexample}{$\lambda$式の正規形への変換}
\[
 (\lambda x . x) (\lambda y . y) \underset{\beta}{\longrightarrow}
 \lambda y . y 
\]
 この右辺は正規形である.
\end{myexample}

正規形にならない$\lambda$式もある.
\begin{myexample}{正規形にならない$\lambda$式}
 $X = \lambda x . (xx) x$のとき,
 \[
  XX \underset{\beta}{\longrightarrow} (XX)X
 \underset{\beta}{\longrightarrow} ((XX)X)X
 \underset{\beta}{\longrightarrow} \cdots
 \]
 どんどん式が長くなっていく方に変換されていく.
\end{myexample}

\section{$\lambda$式による論理の表現}
以下,$[\alpha]$で概念$\alpha$を表す$\lambda$式を意味するものとする.

\begin{eqnarray*}
 \left[true\right] &\equiv& \lambda xy . x \equiv T \\
 \left[false\right] &\equiv& \lambda xy . y \equiv F
\end{eqnarray*}
このように定義すれば,式$P$の真偽によって,
\[
PMN \underset{\beta}{\longrightarrow}
  \left\{
   \begin{array}{l}
    M : Pが真 \\
    N : Pが偽
   \end{array}
  \right.
\]
と,if-then-elseを表せる.更に,
\begin{eqnarray*}
 \left[not\right] &\equiv& \lambda pxy . pyx \footnotemark \\
 \left[and\right] &\equiv& \lambda pq . pqp \footnotemark \\
 \left[or\right] &\equiv& \lambda pq . ppq \footnotemark
\end{eqnarray*}
\setcounter{myfootnote}{\value{footnote}}
\addtocounter{myfootnote}{-2}
\footnotetext[\value{myfootnote}]{$p, q$は真偽値.引数の,「もし$p$が真
なら$x$を,偽なら$y$を返すような式」というのを逆さまにして,「もし$p$が真
なら$y$を,偽なら$x$を返すような式」としている.}
\addtocounter{myfootnote}{1}
\footnotetext[\value{myfootnote}]{もし$p$が真なら$q$の
値.さもなくば$p$の値,すなわち偽.}
\addtocounter{myfootnote}{1}
\footnotetext[\value{myfootnote}]{もし$p$が真なら$p$(真),さもなくば
$q$($q$の真偽に委ねられる).}

\section{$\lambda$式による自然数の表現} \footnote{数を式で表すのですか
ら,偉大な抽象化です.ロマンを感じましょう.}
$\lambda$式による自然数の表現 \footnote{現代的には,自然数は集合によって
定義するのが普通.
\begin{eqnarray*}
 \left[0\right] &\equiv& \{\} \\
 \left[succ(x)\right] &\equiv& \{x\} \cup x
\end{eqnarray*}
$0$を含み$succ$について閉じている集合は存在(公理).自然数は,「そのよう
な集合の共通部分」と定義.
}
は色々知られているが,ここではChurch
numeral \footnote{「数」はnumberだが,「数を表す表記法」はnumeralとい
う.また,「数字」はdigit.}
と呼ばれるものを扱う.その表記法では,自然数$n$を,次のような二引数の高階関数として表現
する.
\begin{description}
 \item[第一引数] 何らかの関数$f$
 \item[第二引数] その関数に渡す値$x$
 \item[結果] $f$を$x$に$n$回適用したもの
\end{description}
すなわち,関数の適用回数で自然数を表していると言える.実際に,
\begin{eqnarray*}
 \left[0\right] &\equiv& \lambda fx . x \footnotemark \\
 \left[1\right] &\equiv& \lambda fx . fx \\
 \left[2\right] &\equiv& \lambda fx . f(fx) \\
 &\cdots& \\
 \left[n\right] &\equiv& \lambda fx . f(f(\cdots (fx) \cdots))
  \hspace{2ex} ただしfはn回適用.
\end{eqnarray*}
\footnotetext{[0]の定義は,[false]の定義を満たすこ
とに注意.すなわち,[0]はFである.これは,すぐ下の
[iszero]の定義で必要になる考えである.}
と表現できる.この表現を取ると,自然数$n$に対して次のような式が定義でき
る.
\begin{eqnarray*}
 \left[iszero\right] &\equiv& \lambda n . n(\lambda x . F)T \footnotemark \\
 \left[succ\right] &\equiv& \lambda nfx . f(nfx) \footnotemark \\
 \left[plus\right] &\equiv& \lambda mnfx . mf(nfx) \footnotemark
\end{eqnarray*}
\setcounter{myfootnote}{\value{footnote}}
\addtocounter{myfootnote}{-2}
\footnotetext[\value{myfootnote}]{$n$が0,すなわち$n$が$F$を表すなら,適用は0回だから$T$を返す.0でなければ,1
回以上適用するので$F$を返す}
\addtocounter{myfootnote}{1}
\footnotetext[\value{myfootnote}]{$f$をもう一度適用すれば,$n$
  の次の数}
\addtocounter{myfootnote}{1}
\footnotetext[\value{myfootnote}]{$m+n$を表すには,xに$f$を$m+n$回適用す
  ればよい.ここでは,$f$を$n$回$x$に適用した結果に,更に$f$を$m$回適用している.}

\section{$\lambda$式によるデータ構造}
$\lambda$式を使えば,あらゆるデータ構造が表現できる.まず,$\lambda$式に
よる\mystrong{pair}の表現を見てみる.pairさえあれば,まずlistが「最初の
要素と残り」という形で再帰的に定義でき,listがあればあらゆるデータ構造が
表現可能である.\footnote{「listがあればあらゆるデータ構造が表現可能」っ
ていうのはひょっとしたら嘘かもしれません(でも確かできたはず).ご指摘大歓
迎です.}

\begin{eqnarray*}
 \left[pair\right] &\equiv& \lambda fsb . bfs \footnotemark \\
 \left[first\right] &\equiv& \lambda p . pT \\
 \left[second\right] &\equiv& \lambda p . pF
\end{eqnarray*}
\footnotetext[\value{myfootnote}]{$f$はfirst,$s$はsecond,$b$は
boolean.}
少々分かりにくいが,次の例でこれが実際にpairを表せていることを確かめる.

\begin{myexample}{データ構造の$\lambda$式による表現の例}
 \[
  \left[pair(A,B)\right] = (\lambda fsb . bfs) \left[A\right] \left[B\right]
 \]
 \begin{eqnarray*}
  \left[first(pair(A,B))\right] &=& (\lambda p . pT) \{(\lambda fsb .bfs)
   \left[A\right] \left[B\right] \} \\
  &=& (\lambda fsb . bfs) \left[A\right] \left[B\right] T \\
  &=& T \left[A\right] \left[B\right] \\
  &=& \left[A\right]
 \end{eqnarray*}
 \begin{eqnarray*}
  \left[second(pair(A,B))\right] &=& (\lambda p . pF) \{(\lambda fsb .bfs)
   \left[A\right] \left[B\right] \} \\
  &=& (\lambda fsb . bfs) \left[A\right] \left[B\right] F \\
  &=& F \left[A\right] \left[B\right] \\
  &=& \left[B\right]
 \end{eqnarray*}
\end{myexample}

\section{再帰呼出し(Recursive call)}
ここまでで,$\lambda$式で何でも表現できるような気がしてきたが,再帰呼出
し(recursive call)は中々難しい.何故ならば,$\lambda$式では関数に名前をつけ
ないので,関数の中からストレートに「自分」を呼出す再帰呼出しはできない.

しかしながら,下の例で示す fixed-point operator というものを$\lambda$式
で表現することにより,可能である.

\begin{myexample}{階乗}
 \[
  f(0) = 1 , f(n+1)  = n \times f(n)
 \]
 で階乗は再帰的に定義できる.これを$\lambda$式で表そうとしてみる.まず,
 一旦引数にする.
 \[
  g = \lambda fn . \left[n = 0 なら1, さもなくば n \times f(n)\right]
 \]
 このような関数$g$は,$\lambda$式として定義可能.この$g$に$f$を引数とし
 て渡したものが,定義したい$f$.

 このように, $f = g(f)$となるような$f$を,$g$の不動点(fixed point)とい
 う.

 従って,不動点を返すような関数が$\lambda$式で書ければよい.実際これは書
 けて,
 \[
  Y \equiv \lambda f . ((\lambda x . f(xx)) (\lambda x . f(xx))) \footnotemark
 \]
 
 この$Y$が不動点を返すことを確かめてみると,\footnote{あまりにも分かりに
 くいんで色付けしました.色付けも規則もへったくれもなく脳内でやったの
 で,気に喰わない人は白黒と思って読んでください.}
 \begin{eqnarray*}
  Yg &=& (\lambda \textcolor{Green}f . ( \textcolor{red}{(\lambda x . \textcolor{Green}{f}(xx))}
   \textcolor{blue}{(\lambda x . \textcolor{Green}{f}(xx))} )) \textcolor{Green}{g} \\
  &=& \textcolor{red}{(\lambda \textcolor{blue}{x} . g(\textcolor{blue}{xx}))} \textcolor{blue}{(\lambda x . g(xx))} \\
  &=& \underline{g\textcolor{DeepPink}{(\lambda \textcolor{Indigo}{x} . g(\textcolor{Indigo}{xx}))} \textcolor{Indigo}{(\lambda x . g(xx))}} \\
  &=& g( \underline{g((\lambda x . g(xx))(\lambda x . g(xx)))}) \\ 
  &=& g(Yg)
 \end{eqnarray*}
 となり,確かに.

 これを用いて階乗を表すと,
 \begin{eqnarray*}
  \left[fact\right] &=& g(Yg) \\
  \left[fact(5)\right] &=& g(Yg) 5 \\
  &=& \left[5 = 0 なら1,さもなくば 5 \times (Yg) 4\right] \\
  &=& 5 \times (Yg)4
 \end{eqnarray*}
\end{myexample}
\footnotetext{これには名前がついていて,Y combinator, fixed-point operator,
 paradoxical operatorとか呼ぶ.}

\section{$\lambda$計算と,帰納的関数やTMは同じ計算能力}
証明は省いてました.

\section{同知性の決定不可能性(Undecidability of equivalence)}
2つの$\lambda$式が同値か否かを決定するアルゴリズムは存在しない.
従って(この論理は飛躍.来週の補完に期待),正規形の存在は決定不能.
