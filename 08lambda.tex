\chapter{$\lambda$計算(Lambda calculus)}
\footnote{$\lambda$計算を実装したプログラミング言語としてはLispがとても有
名です.この授業の勉強で使えとは当然言いません
が,\cite{Gerald_Julie_Harold_和田200002}を読んでみると(プログラム的に)
見える世界が変わってきます.個人的イチオシ.}

\mystrong{$\lambda$計算(lambda calculus)}とは,帰納的関数を数と無関係に
定義する枠組みである.

\section{$\lambda$抽象化(Lambda abstraction)}
計算の手続きに名前をつけることができる.例えば,
\[
 a \rightarrow 1+1
\]
など.ここで更に
\[
 b \rightarrow 1+1
\]
とすると,これらの手続きは名前は違えど,表す手続きの内容は同じである.こ
のような手続きを一緒くたにまとめることを,\mystrong{$\lambda$抽象化(Lambda abstraction)}という.
\footnote{$\lambda$抽象化については授業であまり説明がなく,この辺は自分の理
解で書きました.もしかしたら間違ってるかもしれないので注意.}
\footnote{$\lambda$抽象化を利用すれば,いちいち名前をつけずに手続きを定義で
きます.これはプログラムでは無名関数という形で出てき
て,Lisp,Javascript,その他最近の言語では無名関数が使えます.無名関数の
概念は,次のような感じ($do\_anything\_to\_x$を呼び出す際の第1引数が無名関数
になっています.)
\begin{equation} \label{eq:08koukai}
 do\_anything\_to\_x (\{ x * x \}, x); 
\end{equation}
}

\section{$\lambda$式(Lambda expression)}
\mystrong{$\lambda$式(lambda expression)}は,
\[
\lambda x. 定義
\]
と表記する.
 \footnote{$\lambda x. 定義$の$x$は変数名}
$\lambda x.定義$は,
\[
 「ひとつの引数を取って定義の内容を返す関数」を表す式
\]

\begin{myexample}{$\lambda$式の例}
\[
 f(x) = 3x + 2 \hspace{2ex} g(x) = 3x + 2\hspace{2ex}\cdots 
\]
 などの,「1変数をとり,その変数を3倍した数に2を足した数を返す\mystrong{関数}」は,全て
 \[
  \lambda x. (3x+2)
 \]
 という\mystrong{式}で表せる.
\end{myexample}

\section{高階関数(Higher-order function)}
\mystrong{高階関数(Higher-order function)}とか,関数を引数や結果とする関
数である.\footnote{脚注に書いた$do\_anything\_to\_x$ (式\ref{eq:08koukai})も一種の高階
関数です.}

\begin{myexample}{ある関数を2回適用する関数を高階関数として記述する}
 \[
  g(x) = f(f(x))
 \]
 この$g$と同じ意味を持つ,
 \[
  twice(f)
 \]
 という関数を作りたい.これを$\lambda$式として書くと,
 \[
  twice = \lambda f . (\lambda x . fx)
 \]
 となる.\footnotemark
\end{myexample}
\footnotetext{$\lambda$式の表記法は勉強しておきましょ
う.\mystrong{$\lambda$式で解答するような問題が試験にも出題され
る模様です.}}

\section{カリー化(Currying)}
\mystrong{カリー化(Currying)}とは,複数引数の関数を単一引数で表現するこ
とをいう.

\begin{myexample}{簡単なカリー化}
 \[
  f(x, y, z) = x + y \times z
 \]
 という3変数関数を,
 \[
  f_{x,y} (z) = x + y \times z
 \]
 という風に1引数関数にすることもできる($x,y$を固定した).これがカリー化.

 この$f_{x,y}$は,$\lambda$式では
 \[
  \lambda x . (\lambda y . (\lambda z . (x + y \times z)))
 \]
 と表記できる.
\end{myexample}

\section{$\lambda$式について覚えておきたいこと}
\subsection{$\lambda$式の略記法}
$\lambda$式の基本形は$\lambda 変数名 . 定義$であるが,「概念的に複数引数
を取る$\lambda$式」は,次のように表記できる.
\[
\lambda x_1 x_2 \cdots x_n . 定義 \equiv \lambda x_1 . (\lambda
x_2  (\cdots . (\lambda x_n . 定義))) \footnote{カリー化の逆適用ですね.}
\]

\begin{myexample}{略記法の簡単な例}
\[
 \lambda xyz . (x+y \times z) \equiv  \lambda x . (\lambda y . (\lambda z . (x + y \times z))) 
\]
\end{myexample}


\subsection{関数の適用順}
関数を連続して適用するとき,その適用順は次のとおりである.
\[
 M_1 M_2 M_3 \cdots M_n \equiv (((\cdots (M_1M_2)M_3)\cdots)M_n)
\]
ただし,この適用順を明示的に変更したいときは,括弧を使う.

\begin{myexample}{関数の適用順の例}
\[
 fffx \equiv (((ff)f)x) \footnote{「関数に対して関数を入力しているので変に思
 うかもしれませんが,それが高階関数というものです.」}
\]
 高校で習った「合成関数」を再現したいときは,
 \[
 f(f(fx))
 \]
 という風に括弧でくくります.
\end{myexample}

\subsection{基本的な関数}
\[
 I \equiv \lambda x . x\hspace{2ex};Identity
\]
\[
 K \equiv \lambda xy . x\hspace{2ex};定数値を返す関数 
\]


\section{$\beta$変換($\beta$ reduction)}
式$M$の中の全ての自由な \footnote{$\lambda$式の引数になっている引数
は,$\lambda$式によって束縛されています.つまり,$\lambda$式によってロー
カル変数が作れます.Lispのletですね.胸熱.}$x$を式$N$で置き換えた結果を,
\[
 M[x:=N]
\]
と書く.

\begin{myexample}{略記法の例}
\[
 \lambda x . (x + (\lambda x . x) 5) = \lambda x .M 
\]
とすると,
 \[
 M[x:=3] = 3 + (\lambda x . x)5
 \]
\end{myexample}


この略記法を用いて,以下のように\mystrong{$\beta$変換($\beta$
reduction)}が定義される.\footnote{$\alpha$変換もある.それは,
\[
 \lambda x. x \underset{\alpha}{\longrightarrow} \lambda y . y
\]
のように定義される.}

\begin{description}
 \item[引数渡し]  \mbox{} \\
            $(\lambda x . M) N \underset{\beta}{\longrightarrow} M[x:=N]$ 

 \item[$M \underset{\beta}{\longrightarrow} N$ならば,] \mbox{} \\
            \begin{description}
             \item[定義の中での変換] \mbox{} \\
                        $\lambda x . M \underset{\beta}{\longrightarrow} \lambda x . N$ 

             \item[引数の変換]  \mbox{} \\
                        $ PM \underset{\beta}{\longrightarrow} PN$ 
             \item[関数の変換]  \mbox{} \\
                        $MP \underset{\beta}{\longrightarrow} NP$
            \end{description}
\end{description}


\begin{itemize}
 \item $\beta$変換の対象となる式を$\beta$基($\beta$-redex)という.
 \item 式$P$から有限回の$\beta$変換で,
       \[
        P = P_1 \underset{\beta}{\longrightarrow} P_2 \underset{\beta}{\longrightarrow} \cdots
       \underset{\beta}{\longrightarrow} P_n = Q
       \]
       となるとき,この列$P_1 \cdots P_n$を$\beta$変換列といい,
       \[
        P \underset{\beta}{\longrightarrow} Q
       \]
       と表す.
\end{itemize}

\section{Church-Rosserの定理}
\mystrong{Church-Rosserの定理}とは,次のものである.

\begin{mytheorem}
同じ式から$\beta$変換列で導かれた二つの式からは,必ず$\beta$変換列によっ
て同じ式を導くことができる.
\end{mytheorem}
なお,この性質を合流性(Confluence)と呼ぶこともある.

図
\ref{fig:img/08beta_reduction.eps}は,合流性を表すものである.図中の矢印
は全て$\beta$変換を表す.例えば,2つの黄色い式から,赤い式に「合流」して
いることが見て取れる.

\myfigure{img/08beta_reduction.eps}{$\beta$変換列の合流性}

