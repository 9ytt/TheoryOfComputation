\chapter{データの複雑さ(Data Complexity)}
データの複雑さを表す尺度として,\mystrong{コルモゴロフ複雑性(Kolmogorov
complexity)}がある.これは,記号長を生成するのに必要なプログラムの最小長
である.

例えば,'101010101...10 (1万桁)'は,大きなデータであるが,単純なループで
書けてしまうので,データの複雑さは小さいと言える.

\section{定式化}
プログラム$p$をある計算機$m$上で動作させた出力を$m(p)$とすると,ある記号
列$s$に対して$m(p) = s$となる最短のプログラム$p$の長さは,
\[
 K_m (s) = \min _{p | m(p) = s} \left(length(p)\right)
\]
と表せる.これがコルモゴロフ複雑性である.

\section{性質}
\begin{itemize}
 \item $s$に規則性があれば,$s$を定数として含むプログラムより短いプログ
       ラムで生成可能.
 \item 2つの異なる言語でも,一方で他方のエミュレータを作れるなら,複雑性
       はエミュレータのプログラム長(定数)しか異ならない.
\end{itemize}

この性質は,データ圧縮の可能性に関係している.つまり,コルモゴロフ複雑性
が低いデータは高い圧縮率で圧縮できる.

ただし,
\begin{mytheorem}
 \item コルモゴロフ複雑性は計算不能.
\end{mytheorem}
である(計算不能とは,どのような入力に対しても出力を出すようなアルゴリズ
ムは存在しないということであった).

\begin{myproof}{コルモゴロフ複雑性は計算不能であることの証明}
 どのようなデータ$s$に対しても$K(s)$を導くようなアルゴリズムの存在を仮定
 して,矛盾を導く.

$K(s)$が計算できるのだから,プログラム長$l$のプログラムでは生成できない
 様な最短の記号列を求めることができる.それは,記号列を短い方から系統的
 に生成して,$K(s)$が$l$を超える最初の記号列を返すことにより実現できるは
 ずである.

このようなプログラムを$C(l)$とし,その長さを$L_C$とするとする.この関数
 を引数$l_0$で呼出した結果を返すプログラムを考える.この関数を引数$l_0$
 で呼出した結果を返すようなプログラムを考えると,その長さは
\[
 L_c + \log_2 (l_0) + \alpha
\]
となる.この長さのうち,$\log_2 (l_0)$以外は定数である.従って,$l_0$を
 大きくしていくことにより,いつかは
\[
 l_0 > L_C + \log_2 (l_0) + \alpha
\]
となるはずである.このプログラムは$C(l_0)$,即ち$l_0$の長さのプログラム
 では生成できない記号列を返すはず.それなのに長さが$l_0$未満であるため,
 矛盾である.
\end{myproof}
