\chapter{計算量理論}
\mystrong{計算量理論(Computational Complexity Theory)}は,特定のアルゴリ
ズムの性能解析手法である.

\section{漸近的計算量(Asymptotic complexity)}
\mystrong{漸近的計算量(Asymptotic complexity)}とは,処理対象のデータ量が
増えたときに,計算量はデータ量のどのような関数になるのかを扱う指標である.

\since データ量小さいときは,どうせ計算量も小さいので気にする必要はない.

問題サイズ(問題となるサイズ.例えば,処理対象のデータ量そのもの)を$n$と
する.また,計算量を$T(n)$とする.
通常$T(n)$は単調増加.

\subsection{$\Omega$記法($\Omega$-notation)}
\mystrong{$\Omega$記法($\Omega-notation$)}は,計
算量の漸近的下界を示す.あるアルゴリズムが$\Omega(f(n))$という計算量を持
つとは,
\[
 \exists c \exists n_0 \forall n \geq n_0 \hspace{2ex}(T(n) \geq c f(n))
 \hspace{3ex} (cは定数)
\]
が成立するということである.(参考: 図\ref{fig:img/10omega.eps})

\myfigure{img/10omega.eps}{$\Omega$記法の式のイメージ}

\subsection{$O$記法(big-$O$ notation)}
\mystrong{$O$記法(big-$O$ notation)}は,計算量の上界を示す.$O(f(n))$の計算量とは,
\[
 \exists c \exists n_0 \forall n \geq n_0 \hspace{2ex}(T(n) \leq c f(n))
\]
が成立するということである.

\subsection{$\Theta$記法($\Theta$-notation)}
\mystrong{$\Theta$記法($\Theta$-notation)}により,
$f$と$g$の上界と下界の関数形が一致する(上界と下界が定数系の違い)とき,すなわち
\[
 f(n) = O(g(n)) \hspace{2ex} かつ \hspace{2ex} f(n) = \Omega (g(n))
\]
のとき,
\[
 f(n) = \Theta (g(n))
\]
と表記する.

それぞれの$c, n_0$は別々に考えて良い.


\section{入力データと計算量}
単純でないアルゴリズムの計算量指標は,入力データの量だけでなく,質によっ
て計算量が異なる場合がある.

\begin{myexample}{クイックソート}
 クイックソートは,ピボットの位置によって,計算量が最悪で$T(n) = c n^2$
 になってしまう.
\end{myexample}

\subsection{最悪の場合の計算量(Worst-case complexity)}
\mystrong{最悪の場合の計算量(Worst-case complexity)}は,「どんなデータに対
しても,最悪でこの計算量を保証しますよ」というようなものである.計算資源
に絶対的制約がある際には不可欠である.

\begin{myexample}{リアルタイム制御}
 原子炉や証券取引を制御するプログラムを考えてみると,「この計算は普通1ms
 で終わりますけど,ひょっとしたら1s掛かる時もあります.まぁ滅多にそんな
 ことは起きませんけどね.」は通用しない.最悪の場合の計算量を見積る必要
 がある.
\end{myexample}

\subsection{平均計算量(Average-case complexity)}
\mystrong{平均計算量(Average-case complexity)}は,様々な入力データに要す
る計算量の平均である.この「様々な入力データ」には,分布の仮定が必要.そ
の仮定に対してのみ平均計算量が出せる.

\subsection{償却計算量(Amortized complexity)}
\mystrong{償却計算量(Amortized complexity)}は,複数操作\footnote{複数操
作とは,例えば,AVN木での「木のバランスを方操作・挿入」をまとめたも
の.AVN木は,木のバランスが前提のデータ構造なので,これらをまとめて考え
る価値がある.}の系列の総計算
量.\ref{sec:償却計算量}に詳細を記述する.


\section{計算量クラス}
\subsection{クラスP(Polymial time)}
決定性チューリングマシン(やそれと等価なもの)によって,問題サイズの多項式関数で表すことのできる計
算量のアルゴリズムが存在するような問題のクラスのことを,\mystrong{クラスP}と呼ぶ.

\subsection{クラスNP(Non-deterministic polynomial time)}
非決定性チューリングマシン(やそれと等価なもの)によって,問題サイズの多項式関数で表すことのできる計
算量のアルゴリズムが存在するような問題のクラスのことを,\mystrong{クラスNP}と呼ぶ.

決定性チューリングマシンは,非決定性チューリングマシンの特別な場合に過ぎ
ないので,
\[
 NP \subseteq P
\]
である.

\subsection{NP困難(NP-hard)}
どんなNPの問題でも,多項式時間の計算量でその問題に帰着できるような問題の
クラスのことを,\mystrong{NP困難}な問題のクラスと呼ぶ.

\subsection{NP完全(NP-complete)}
NPに属すNP-hardな問題のクラスは,\mystrong{NP完全(NP-complete)}な
問題のクラスである.すなわち,
\[
 NP-complete = NP \cap NP-hard
\]

\subsection{P $\neq$ NP ?}
P $\neq$ NPであるという予想が立っているが,これは未解決問題である.

\subsection{Pに入るか否か不明の問題}
P $\neq$ NP を仮定すると,NP-completeならPには入らない.一方で,
\begin{itemize}
 \item 素因数分解
 \item グラフ同型問題
\end{itemize}
などは,Pに入るか否か不明の問題である.

\subsection{Time-Hierarchy Theorem}
どのような時間計算量の問題クラスに対しても,それより複雑な計算量の問題が
存在する.

\subsection{Space-Hierarchy Theorem}
(板書なし)

\section{多倍長カウンタの例}

\begin{myexample}{多倍長カウンタ}
1 word(通常32bit)で表現しきれない数値を整数の配列で表現することがある.整数の配列を
 $a[i]$とすると,
 \[
  n = \Sigma^{n-1}_{k=0} \{a[k] \times 2^k\}
 \]

 これを用いた,下記のアルゴリズムincrementの計算量を考える.

 \begin{description}
 
  \item[worst-case] \mbox{} \\
             carryはいくらでも出せるので,最悪ではビット数に比例.
             例えば,0111から1000にカウントアップするときに最悪.
  \item[average-case]\mbox{} \\
             $2^n$までカウントすると,合計では$2^n$のオーダー1回あたりの
             平均計算量は$O(1)$.ちゃんと計算すると,平均計算量は2と求め
             られるらしい.
 \end{description}

\end{myexample}

\begin{mypre}{アルゴリズムincrement}
void increment(int a[])
{
    int k;
    for (k = 0; a[k] == 1; k++)
        a[k] = 0;
    a[k] = 1;
}
\end{mypre}

確かに最悪の場合ビット数に比例した計算量だが,最悪の場合はカウントアップ
を初めてからかなり後にしか起きない.\mystrong{平均計算量では,各状態での操
作の手間について何も保証できていない}ので,このようなミスマッチが生じる.

最初の方の操作の手間は,ビット数$n$に依存しないことを表現できる枠組みが
欲しい.それが下記の償却計算量である.


\section{償却計算量(Amortized complexity)} \label{sec:償却計算量}
「償却」に込められた意味は,「実質的に意味のある値」みたいな感じ.減価償
却も同じ用法.\footnote{ただし,減価償却は前払いであるのに対し,カウント
アップの例は後払い(値が0に近いうちはあまりコストが掛からない).この点
で,先生は「あまり良い訳ではない」と仰ってました.どうでもい(ry}

この考えのもとで,先程のアルゴリズムincrementを見直してみる.

\begin{enumerate}
 \item たくさん繰り上げが生じると,その後しばらくは繰り上げが出にくい.
 \item たくさんの繰り上げを「投資」と思えば,繰り上げにより生じる下位の
       ビット0の連続は「資産」.コストはこの資産を利用するときに払う.
\end{enumerate}

また,「引当金」\footnote{気になったらググッてください}と同じ意味での引
当ての考え方で見てみる.

\begin{enumerate}
 \item 実際の「コスト」は$0 \rightarrow 1, \, 1 \rightarrow 0$が単位.
 \item 1のビットを作ると将来carryを生む原因になるので,$0 \rightarrow 1$
       の反転のコストを2と考える.実際のコスト1に引当分の1を加えたという
       計算.
 \item $1 \rightarrow 0$は引当済み.つまり,コストは$0 \rightarrow 1$に
       なったときに計上済みなので,$1 \rightarrow 0$のコストは0と考える.
\end{enumerate}

これによると,カウントアップ時に行う操作のコストは,
\begin{itemize}
 \item $[1 \rightarrow 0 のコスト] \times [キャリービット数] = 0$
 \item $[0 \rightarrow 1 のコスト] \times [1ビットだけ] = 2$ 
\end{itemize}
(毎回のカウントアップで,$0 \rightarrow 1$になるビットはいつもひとつであ
ることに注意.)従って,カウントアップのコストは常に2である.


\begin{table}
\begin{center}
 \caption{カウントアップの計算量}
% BEGIN RECEIVE ORGTBL table1
\begin{tabular}{|l|r|r|r|r|r|}
\hline
操作 &  & 実際の手間 & コスト累計 & 引当(償却)計算量 & 残 \\
\hline
0 $\rightarrow$ 1 & 0001 & 1 & 1 & 2 & 1 \\
1 $\rightarrow$ 2 & 0010 & 2 & 3 & 4 & 1 \\
2 $\rightarrow$ 3 & 0011 & 1 & 4 & 6 & 2 \\
3 $\rightarrow$ 4 & 0100 & 3 & 7 & 8 & 1 \\
4 $\rightarrow$ 5 & 0101 & 1 & 8 & 10 & 2 \\
5 $\rightarrow$ 6 & 0110 & 2 & 10 & 12 & 2 \\
6 $\rightarrow$ 7 & 0111 & 1 & 11 & 14 & 3 \\
7 $\rightarrow$ 8 & 1000 & 4 & 15 & 16 & 1 \\
\end{tabular}
% END RECEIVE ORGTBL table1
\end{center}
\end{table}

\iffalse
#+ORGTBL: SEND table1 orgtbl-to-latex
|--------+------+------------+------------+------------------+----|
| 操作   |      | 実際の手間 | コスト累計 | 引当(償却)計算量 | 残 |
|--------+------+------------+------------+------------------+----|
| 0 -> 1 | 0001 |          1 |          1 |                2 |  1 |
| 1 -> 2 | 0010 |          2 |          3 |                4 |  1 |
| 2 -> 3 | 0011 |          1 |          4 |                6 |  2 |
| 3 -> 4 | 0100 |          3 |          7 |                8 |  1 |
| 4 -> 5 | 0101 |          1 |          8 |               10 |  2 |
| 5 -> 6 | 0110 |          2 |         10 |               12 |  2 |
| 6 -> 7 | 0111 |          1 |         11 |               14 |  3 |
| 7 -> 8 | 1000 |          4 |         15 |               16 |  1 |
\fi

\begin{myexample}{配列の拡張}
 配列は,予めサイズを決めるのが難しい.従って,最初は小さなサイズで確保
 して,必要に応じて拡張していくことが多い.しかし,配列では連続領域が必
 要なので,拡張時に連続領域が足りなくなったら,別の連続領域をとれる場所
 に現在の中身をコピーしなければならない.この計算量を考えてみる.

 サイズが足りなくなったら倍に拡張するとする.
 拡張して$2^n$要素とした後,次の拡張までにまた$2^{n-1}$要素を追加できる.こ
 の間に,次回の拡張に要する手間を引当てていると考えられる.

定数$a, c$を使うと,拡張時の割付けの手間は$2^{n+1}a$,コピーの手間は$2^n
 c$なので,引き当てるべき計算量は,
\[
 \frac{2^{n+1} a + 2^n c}{2^{n-1}} = 4a + 2c
\]
である.つまり,配列は拡張時には確かに計算量は大きくなるが,1回の操作あ
 たりの手間は定数オーダであると言える.
\end{myexample}



