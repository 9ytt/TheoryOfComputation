\chapter{組合せ論理 (Combinatory Logic)}
\mystrong{組合せ論理(Combinatory Logic)}は,計算モデルの一つであり,関数
抽象化をせずに,\footnote{$\lambda$計算と対照的} 構造の変換だけで計算す
る体系をもつ.

\section{組合せ項(Combinatory Term)}
\mystrong{組合せ項(Combinatory term)}は,以下のいずれかである.

\begin{itemize}
 \item 変数
 \item プリミティブ: S, K, I combinator
 \item 組 $(E_1 \, E_2)$
\end{itemize}

\section{演算規則}
\subsection{プリミティブの役割}
プリミティブS, K, Iは,\mystrong{簡約化(reduction)}を行う.
それぞれの行う演算は,以下の通りである.

\begin{itemize}
 \item $(I \, x) \Rightarrow x$
 \item $((S \, x) \, y) \Rightarrow x$ 
 \item $(((S \, x) \, y) \, z) \Rightarrow ((x \, z)(y \, z))$
\end{itemize}

この規則は,二分木構造で見ると多少覚えやすい.

\begin{center}
\begin{minipage}{7ex}
 \tree{}
  \leaf{$I$}
  \leaf{$x$}
 \endtree
\end{minipage}
\begin{minipage}{5ex}
 $\rightarrow x$
\end{minipage}
\end{center}

\begin{center}
\begin{minipage}{9ex}
 \tree{}
  \subtree{}
   \leaf{$K$}
   \leaf{$x$}
  \endsubtree
  \leaf{$y$}
 \endtree
\end{minipage}
\begin{minipage}{5ex}
 $\rightarrow x$
\end{minipage}
\end{center}

\begin{center}
\begin{minipage}{9ex}
 \tree{}
  \subtree{}
   \subtree{}
    \leaf{$S$}
    \leaf{$x$}
   \endsubtree
   \leaf{$y$}
  \endsubtree
  \leaf{$z$}
 \endtree
\end{minipage}
\begin{minipage}{4ex}
 $\rightarrow$
\end{minipage}
\begin{minipage}{9ex}
 \tree{}
   \subtree{}
     \leaf{$x$}
     \leaf{$z$}
   \endsubtree
   \subtree{}
     \leaf{$y$}
     \leaf{$z$}
   \endsubtree
 \endtree
\end{minipage}
\end{center}

\subsection{左方優先}
左方優先性を持つ.
\[
 (A\, B \, C \, D) \rightarrow (((A \, B)C)D)
\]

\section{Iなんていらないわ}
$I = SKK$なので,$I$がなくても体系が構築できる.\footnote{Emacs使いの皆
さんにはお馴染みのskkは,ここから名前をもじったそうです.}
実際,

\begin{eqnarray*}
 ((SKK) x) &\rightarrow& ((K x)(K x)) \footnotemark \\
 &\rightarrow& (K \, x \, (K \, x)) \\
 &\rightarrow& x
\end{eqnarray*}
\footnotetext{右側の$(K \, x)$は,$K$がオペレータを2個取らないからといって,別にilligalな訳ではない.あくまでも,$K$の
持つ計算規則によって簡約化できないだけ.}

なお,$I$を作るのは$SKK$だけでなく,例えば$I = SKS$でもある.

\section{組合せ論理は$\lambda$計算と等価}
組合せ論理は,以下の点で$\lambda$計算と等価である.
\begin{itemize}
 \item どんな$\lambda$式に対しても等価な組合せ項がある.
 \item どんな組合せ項に対しても等価な$\lambda$式がある.
\end{itemize}

実際,次のように$\lambda$式と組合せ論理式を対応させると,任意の$\lambda$
式を組合せ論理式に変換でき,またその逆もできることが分かる.

($\lambda$式$L$から組合せ項$c$への変換を$T[L] \rightarrow c$という風に表記
する.)

\begin{enumerate}
 \item $T[x] \rightarrow x$
 \item $T[FX] \rightarrow (T[F] \, T[X])$ 
 \item $T[\lambda x . E] \rightarrow (K \, T[E])$ \\
       ただし,$E$の中に$x$が出現しないとき.
 \item $T[\lambda x . x] \rightarrow I$
 \item $T[\lambda x . \lambda y . E] \rightarrow T[\lambda x . T[\lambda
       y . E]]$ \footnote{ここにも適用条件はあるはずですが,先生が忘れて
       らっしゃいました.}
 \item $T[\lambda x . (F \, Y)] \rightarrow (S \, T[\lambda x . F] \,
       T[\lambda x . Y])$
\end{enumerate}

\begin{myexample}{変換規則適用の例}
$\lambda x . \lambda y . (y \, x)$を,組合せ論理式に変換する.
\begin{eqnarray*}
  T[\lambda x . \lambda y . (y \, x)] &\rightarrow& T[\lambda x
 . T[\lambda y . (y \, x)]] \hspace{2ex} \since 5 \\
 &\rightarrow& T[\lambda x . (S \, T[\lambda y . y] \, T[\lambda y
  . x])] \hspace{2ex} \since 6 \\
 &\rightarrow& T[\lambda x . (S \, I \, (K \, T[x]))] \hspace{2ex} \since 3, 4 \\
 &\rightarrow& T[\lambda x . (S \, I \, (K \, x))] \hspace{2ex} \since 1 \\
 &\rightarrow& (S \, T[\lambda x . (S \, I)] \, T[\lambda x . (K x)])
  \hspace{2ex} \since 6 \\
 &\rightarrow& (S \, (K \, T[(S \, I)]) \, T[\lambda x . (K x)]) \hspace{2ex}
 \since 3 \\
 &\rightarrow& (S \, (K \, (S \, I)) \, (S \, T[\lambda x . K] \,
 T[\lambda x . x])) \hspace{2ex}
 \since 1, 6 \\
 &\rightarrow& (S (K (S \, I)))(S \, (K \, K) \, I) \hspace{2ex} \since
 3, 4
\end{eqnarray*}
 逆変換も見てみる.
 \begin{eqnarray*}
  (S (K (S \, I))) (S (K \, K) I )\, x\, y &\rightarrow& ((K (S\, I)) \,
   x) ((S (K \, K)I) x) \, y \\
  &\rightarrow& (S \, I) (((K \, K) x) (I \, x)) \, y \\
  &\rightarrow& (S \, I)(K \, x) \, y \\
  &\rightarrow& (I \, y)((K \, x) \, y) \\
  &\rightarrow& y \, x
 \end{eqnarray*}
\end{myexample}

\section{真偽値の表現}
$P$を,$Pxy$が真なら$x$,偽なら$y$となるように作る.
真を表す記号を$T$,偽を表す記号を$F$とすると,それぞれ以下のように組合せ
論理で構成できる.
\begin{eqnarray*}
 &T& = K \\
 &F& = (K \, I)
\end{eqnarray*}

\begin{myproof}{$T$,$F$が組合せ論理で構成できる証明}
\begin{eqnarray*}
 (T \, x \, y) &=& (K \, x \, y) = x \\
 (F \, x \, y) &=& (K \, I \, x \, y) = (I \, y) = y
\end{eqnarray*}
\end{myproof}

\section{組合せ論理の性質}
$\lambda$式と組合せ論理は等価なので,組合せ論理は当然$\lambda$式と同様な
性質を持つ.

\begin{itemize}
 \item これ以上簡約化できない項があれば,それを\mystrong{正規形}と呼ぶ.
 \item 正規形の存在は決定不能.
 \item 項のどの部分から簡約化しても,更に簡約化を進めると同じ項を得られ
       る(\mystrong{合流性}).図\ref{fig:img/08beta_reduction.eps}を参照.
\end{itemize}

\begin{myexample}{正規形を持たない組合せ論理式}
\begin{eqnarray*}
 (S \, I \, I \, (S \, I \, I)) &\rightarrow& (I (S \, I \, I) (I \, (S \,
  I \, I))) \\
 &\rightarrow& (S \, I \, I \, (S \, I \, I))
\end{eqnarray*}
 この項は正規形を持たない.
\end{myexample}


