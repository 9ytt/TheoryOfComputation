%#!platex main.tex

\section{有限オートマトン(Finite automaton, FA)}
\mystrong{有限オートマトン}  \footnote{有限状態オート
 マトン(finite state automaton)と呼ぶこともある} (以下,\mystrong{FA})とは,有限の記憶を持つ計
算機構の一種である.より具体的には,ある入力列を走査し,その入力列を受理
するかを判定するものと言える.

\myfigure[width=60truemm]{img/01FA.jpg}{FAを表す遷移図}

\begin{myexample}{簡単なFA}
 \figurename \ref{fig:img/01FA.jpg}は,FAを表す一つの例である.このよう
 に,FAは\mystrong{状態遷移図}で表せる.まずは,このFAの動作の概略を説
 明する.
\begin{enumerate}
 \item FAは,開始状態(initial state)で開始する.今回の開始状態は$S_1$.
 \item 入力の1ビット目により,次に遷移する状態が変わる.0だったら
       $S_3$に,1だったら$S_2$に移る.
 \item 同様に2ビット目,$\cdots$と入力列を読んでいき,読み終わった時点の
       状態,すなわち最終状態(final state)が$S_3$であれば,このFAは入力
       列を\mystrong{受理(accept)}したといい,$S_3$でなければ受理しなかったとい
       う.
\end{enumerate}

 ここで,実際に入力列$0100$を入れたときの動作を追ってみる.
\begin{enumerate}
 \item 開始状態$S_0$へ.
 \item 0を受け取り,$S_3$へ.
 \item 1を受け取り,$S_3$へ.
 \item 0を受け取り,$S_2$へ.
 \item 0を受け取り,$S_2$へ.
 \item 入力列を読み終わったが,$S_3$でなく$S_2$にいるので,入力列$0100$
       は受理しない.
\end{enumerate}
\end{myexample}

ここで一度,\ref{sec:計算の能力}の記述を確認し,(FAに限らず)オートマトン
の重要性を確認して欲しい.

\subsection{FAの定式化} \label{FAの定式化}
FAの能力,動作などを真面目に議論しようとすると,どうしても定式化する必要
が出てくる.以下のように定式化されることが多い.

\begin{description}
 \item[$M = (Q, \Sigma, \delta, q_0, F)$] \mbox{} \\
            あるFAを表す式.
            
 \item[$Q$] \mbox{} \\
            状態の有限集合.上の例で言うと,$S_1, S_2, S_3$
 \item[$\Sigma$] \mbox{} \\
            入力記号の有限集合.上の例で言うと,$0, 1$
 \item[$q_0$] \mbox{} \\
            初期状態 $\in Q$.上の例で言うと,$S_0$
 \item[$F$] \mbox{} \\
            最終状態の集合 $\subset Q$.ここで,最終状態は1つでなくとも
            良いことに注意.上の例で言うと,$S_2$のみ.
 \item[$\delta$] \mbox{} \\
            遷移関数(transition fucntion).数式でいうと,
            \begin{equation}
             \delta : \hspace{2ex} Q \times \Sigma \rightarrow Q
              \label{eq:02transition_function}
            \end{equation}
            ここで,$\times$記号は直積\footnote{Wikipedia\cite{wikipedia}
            が意外とわかりやすいので,興味があれば調べてみてください.}
            を表す.すなわち,
            \[
             Q \times \Sigma = \{(q, s) | q \in Q, s \in \Sigma\}
            \]
            である.したがって,式\ref{eq:02transition_function}の表すと
            ころは,「ある状態$q$において,次の入力記号が$s$であった場
            合,次の状態は$Q$に含まれるいずれかである.\footnote{ここで
            「いずれか」とはいえど,一意に定まっている.例えば,「状態
            $q_1$で入力$a$を受け取ったら,状態$q_3$に移る.ただし,$q_i
            \in Q, \, a \in S$」ということ.下記のNFAの話を読むと気になる
            ところだと思うので.}といったもので
            ある.
\end{description}

\subsection{オートマトンの動作} \label{オートマトンの動作}
以下の動作は,FAのみならず,オートマトン一般に成り立つことに注意.

\begin{itemize}
 \item 初期状態$q_0$から初めて,入力1記号毎に状態遷移.
 \item 「入力が終わったときに最終状態の1つにある」$\Leftrightarrow$ 「オー
       トマトンは
       入力を受理した」
 \item オートマトンを定義すると,それが受理する入力列の集合(\mystrong{オー
       トマトンの受理言語})が決まる.
\end{itemize}

特に,FAの受理言語を\mystrong{正則集合(regular set)}という.

\section{FAと正規表現の関係}
FAは計算機構であり,正規表現は文法であるが,どちらも言語を与える(前者は
受理言語により).両者の与える言語について,以下の定理が成り立つ.

\begin{itemize}
 \item FAの受理言語は,正規表現で表せる.
 \item 正規表現が表す言語を受理するFAを構成できる.
\end{itemize}

この定理は授業中にも証明しておりません.気になる人は,「有限オートマトン
正規表現 証明」とかでググると結構出てきます.気にならない人は「正規表現
はFAと同程度の表現力なのか」と覚えておけば良いと思います.

\section{非決定性有限オートマトン(Non-deterministic finite automaton, NFA)}
\mystrong{非決定性有限オートマトン}(以下,\mystrong{NFA})は,同じ状態に
あり,同じ入力記号を受け取る時でも,次の状態が一意には決まらないようなFA
である.

\subsection{NFAの定式化}
基本的に,\ref{FAの定式化}でしたのと同じ定式化ができる.ただし,遷移関数
は異なり,以下のように定式化される.
\begin{equation}
 \delta : \hspace{2ex} Q \times \Sigma \rightarrow 2^Q
  \label{eq:NFA_transition_function}
\end{equation}
ここで,$2^X$は,冪集合(power set)を表す記号であ
る.\footnote{\cite{power_set}にそう書いてありました.}
$2^X$は,すべての「集合$X$の部分集合」を元として集めた集合系(集
合族)のことである.\footnote{\cite{power_set}にそう書いてありました.}

したがって,式\ref{eq:NFA_transition_function}の意味するところは,「$Q$
中の状態にいるとき,$\Sigma$上の記号を受け取ると,$Q$上のいくつかの状態
のうちの1つに移る」といったところだろうか.

\subsection{NFAの動作}
基本的には\ref{オートマトンの動作}に書いたオートマトンの動作と同様.ただ
し,
\[
 「NFAが受理する記号列」\Leftrightarrow「入力終了時にとり得る状態のうち
 に,最終状態に含まれるものが存在する」
\]
という点が異なる.この定義は中々複雑に感じるので,以下で細かく説明す
る.

\begin{description}
 \item[\mystrong{「入力終了時にとり得る状態」って??}] \mbox{} \\
NFAは,入力を1個取
            るごとに,何通りかの遷移の可能性を持つ.従って,入力終了時の
            状態は,一意に定まらない.
 \item[\mystrong{「最終状態に含まれるものが存在する」って??}] \mbox{} \\
NFAの定義に
            よると,最終状態は複数あって良い.繰り返しになるが,FAでも最
            終状態は複数会って良い.
\end{description}

\myfigure[width=60truemm]{img/02NFA.jpg}{NFAを表す状態遷移図の例}

\begin{myexample}{簡単なNFA} \myexamplelabel{02NFAex}
 図\ref{fig:img/02NFA.jpg}は,簡単なNFAの例を示している.\footnote{この図で
 は,状態$B$において入力1をとった場合の動作などが明示されていません.そ
 の時は状態遷移しない,すなわち,「同じ状態に遷移する」と考えて良いで
 しょう.なにせネットから引っ張ってきた図ですので :-)}NFAならではの特徴を
 以下に指摘する.
 \begin{itemize}
  \item 状態と入力が決まっても,次状態が一意には決まらない場合がある.例
        えば,状態$D$にいたとき,入力$1$を受け取った場合,次状態としては
        $B$も$C$もあり得る.
 \end{itemize}

また,
\begin{itemize}
 \item 終了状態が複数ある.$C, D$が終了状態である.
\end{itemize}
ことにも注意.

 このNFAが,入力$001$を受理するかを考えてみる.
\begin{enumerate}
 \item 開始状態は$A$.
 \item 1ビット目の入力0をとると,$B$にも$C$にも移れる.いま,$B$に移るとする.
       \label{02NFAex2}
 \item 2ビット目の入力0をとり,$D$に移る.
 \item 3ビット目の入力1をとると,$B$にも$C$にも移れる.いま,$B$に移ると
       する. \label{02NFAex4}
 \item 入力列を読み終わったが,最終状態の集合($C, D$)の元にいないので,
       とりあえず失敗. \label{02NFAex5}
 \item 気を取り直し,\ref{02NFAex4}で$B$でなく$C$に移る.
 \item 入力列を読み終わり,最終状態の集合の元$C$にいるので,\mystrong{こ
       のNFAは入力$001$を受理する}と言える.
\end{enumerate}
 もちろん,\ref{02NFAex5}で失敗が分かったとき,\ref{02NFAex4}でなく
 \ref{02NFAex2}に戻っても良い.いずれにせよ,入力が読み終わった時点で
 いずれかの最終状態にいられるようなパスを1つ見つけてしまえば,NFAは入力
 を受理すると言える.
\end{myexample}

例\myexampleref{02NFAex}により,何となく掴みどころのなかったNFAを分かって頂けたかと思う.

ちなみに,NFAの「失敗したら分かれ道に戻る」という動作は,プログラミング
手法でいうbacktrackに対応する.さらにプログラミング的な話にすると,別に
backtrackを使わずとも,分かれ道(=分岐)が発生した時点で,スレッドなりプロ
セスなりを生成し,並列に入力を取っていってテストしても良い.

\section{NFAとFAの等価性}
ここでは,NFAとFAの等価性について説明する.もちろん「等価」といって
も,NFAとFAは全く同じマシンであるということを主張しているわけではな
い.\mystrong{NFAとFAの受理言語の集合が一致する}という点において等価とい
うことだ.すなわち,
\begin{mytheorem} \label{02fa_to_nfa}
どんなFAについても,同じ受理集合を持つNFAが作れる.
\end{mytheorem}

\begin{mytheorem} \label{02nfa_to_fa}
 どんなNFAについても,同じ受理集合を持つFAが作れる.
\end{mytheorem}
の両方が成り立つということだ.これを以下に示す.

\begin{myproof}{定理\ref{02fa_to_nfa}の証明}
これは,$FA \subset NFA$なので簡単.

 まず,与えられたFA(given FA と表記す
 る)と全く同じ遷移状態を,構成するNFA(obj NFA と表記する)に持
 たせる.一般にはNFAがある状態にいてある入力を受けるとき,遷移状態は1つ
 に定まらないが,今回はgiven FAと全く同じ遷移状態をobj NFAに持たせる.

 更に,obj NFAにはgiven FAと同
 じ最終状態を持たせる.

以上の手順により,obj NFAはgiven NFAと全く同じものになり,すなわち同じ受
 理集合を持つことが示された.
\end{myproof}

\begin{myproof}{定理\ref{02nfa_to_fa}の証明}
 \footnotemark

与えられたNFAを$M = (Q, \Sigma, \delta, q_0, F)$ \footnotemark
 とするとき,対応するFAとして,$M' = (Q', \Sigma, \delta' , q_0, F')$を以下の
 ように構成すれば良い.
 \begin{eqnarray*}
  Q' &=& 2^Q \hspace{2ex} (Qが有限だから2^Qも有限に納まる)\\
  q_0 ' &=& q_0\\
  F' &=& \{S \subset Q | F \cap S \neq \phi \} \footnotemark
 \end{eqnarray*}
 \[
  \delta ' : 2^Q \times \Sigma \rightarrow 2^Q
 \]
 つまり,
\[
 \delta ' (\{q_1, q_2, \cdots , q_n\}, a) = \cup^{n}_{i = 1} \delta (q_i , a) 
\]
  ここで,$\{q_1, q_2, \cdots , q_n\}$は$2^Q$の部分集合, $a$は$\Sigma$ の元,$q_i$は$2^Q$
   の元.

 以下略. \footnotemark
\end{myproof}
 % minipage環境中で脚注番号を自動振りするための知恵
\setcounter{myfootnote}{\value{footnote}}
\addtocounter{myfootnote}{-3}
\footnotetext[\value{myfootnote}]{実は証明になってません.完全な証明を知りた
ければ,\cite{John_Jeffrey_Rajeev_野崎_町田_高橋_山崎200304}のp.29あたり
に載ってます.}
\addtocounter{myfootnote}{1}
\footnotetext[\value{myfootnote}]{引数は左から,状態集合,入力記号の集合,遷移関数,開始状態,終了状態の集合,でした.}
\addtocounter{myfootnote}{1}
\footnotetext[\value{myfootnote}]{ベン図で考えるとわ
   かりやすい.構成するFAの最終状態の中に,与えられたNFAの最終状態である
   ものが存在すれば良い.これも数式で書くと,$$\exists x (x \in F \wedge x
   \in S)$$ってこと.}
\addtocounter{myfootnote}{1}
\footnotetext[\value{myfootnote}]{先生もそんなに厳密性を気にされる方じゃなさ
そうです:-)}

\section{FA,NFAの限界}
ここでは,FAとNFAに共通する限界について説明する.

どちらも状態$Q$は有限であり,それはつまり記憶が有限ということ.従って,
有限の記憶で処理できない計算は,FAやNFAには処理できないということである.

\myfigure{img/02limitsFA.eps}{入力列の0/1の個数が同数かを判定するFA}

\begin{myexample}{入力列の0/1の個数が同数かを判定するFAは構成できない} \myexamplelabel{02FA01}
 図\ref{fig:img/02limitsFA.eps}は,入力列の0/1の個数が同数かを判定するFA
 を表している.

 開始状態が$q_0$で,1を受け取ると$q_1$に,0を受け取ると
 $q_{-1}$に移る.つまり,各状態は,
 \[
  q_{[受け取った1の個数] - [受け取った0の個数]}
 \]
 となる.

 終了状態も$q_0$であるので,受理言語は,「0,1の個数が同数である0/1の列」
 となる.

いま,遷移図の一番左の状態を$q_l$,一番右の状態を$q_r$としよう.FAにおい
 て,状態数は有限なので,必ずこのような状態は存在する.

もし,入力列がある程度短く,
 \[
  |入力列| \leq min\{-l, r\}
 \]
 であるとすると,この入力列を判定することはできるが,入力列が長く,
 \[
  |入力列| > min\{-l, r\}
 \]
 となるとき,状態が$q_l$もしくは$q_r$に達し,更に右か左に動くことができ
 なくなる可能性がある(例: $q_{-1}, q_0, q_1$の3状態から成るFAに,入力列
 $11$を与えたとき).そのような長い入力列に対しては,FAが構成できていない
 ことになる.

そして,FAに与える入力の長さに制限はなく,無限長でも良いので,全ての入力
 列を判定できるFAは存在しない.

 例えば,入力列が$110110\cdots (無限に続く)$では,どれだけ(有限の範囲で)
 状態の多いFAでも対処できない.
\end{myexample}

