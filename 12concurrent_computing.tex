\chapter{並行計算のモデル}
コンピュータで並行というと,次の2つの意味を指す場合がある.
\begin{description}

 \item[Concurrency] \mbox{} \\
            論理的並列性を表す.前後関係が定義されていない2つの計算につ
            いての言葉.論理的正当性に関係する
 \item[Parallelism] \mbox{} \\
            物理的並列性を表す.実際に並列に動作しているかどうか.性能に
            関係する.
\end{description}

Concurrencyが保証されなければParallelismがあっても意味がないことに注意.
本章では,前者のConcurrencyについて議論する.

\section{ペトリネット(Petri Net)}
\mystrong{ペトリネット(Petri Net)}は,並行システムの古典的モデル.並行シ
ステムを有向二部グラフで表現する.Carl Adams Petriの提唱である.

\subsection{構成要素}
\myfigure{img/12petri.eps}{ペトリネットの構成要素}

\begin{description}
 \item[トークン(token)] \mbox{} \\
            処理のアクティビティを表す.例えば,プロセスやスレッド.ペト
            リネットの中をトークンが動き回る.

 \item[場所(place)] \mbox{} \\
            処理の状態を表す.

 \item[遷移(transition)] \mbox{} \\
            場所から場所へ移る際の中継点.

 \item[辺] \mbox{} \\
            「場所 $\rightarrow$ 遷移」,または「遷移 $\rightarrow$ 場所」
            を表す繋ぎ目.「場所 $\rightarrow$ 場所」,或いは「遷移
            $\rightarrow$ 遷移」がないのが有向''二部''グラフである所以.
\end{description}

\subsection{動作規則}

\myfigure{img/12petri_join.eps}{待ち合わせ(join)}
\myfigure{img/12petri_fork.eps}{分岐(fork)}
\myfigure{img/12petri_nondeterministic.eps}{非決定性}

\begin{enumerate}
 \item ある遷移に入る全ての辺の視点にトークンがあれば,その遷移が発動可
       能になる.(join) (図\ref{fig:img/12petri_join.eps})
 \item 遷移が発動すると,遷移から出る辺の終点の場所全てにトークンがひと
       つ増える.遷移に入る辺の視点のトークンはひとつ減る.(fork) (図
       \ref{fig:img/12petri_fork.eps})
 \item 可能な遷移が複数ある時,どの遷移が発動するかは非決定的.(図\ref{fig:img/12petri_nondeterministic.eps})
\end{enumerate}

\subsection{並行版の有限オートマトンとしての見方}
ペトリネットは,有限状態オートマトンの並行版としても見られる.逐次の
state machineは,次の特徴を持つペトリネットとモデル化できる.
\begin{itemize}
 \item 全ての遷移が単一の入出力
 \item トークンが増減しない
 \item 初期状態でトークンはひとつ
\end{itemize}


\section{プロセス計算(Process Calculus)} \footnote{この節については結構
説明が浅かったです.残念.}
並行計算を抽象化して代数的に扱う枠組みの総称.独立したプロセスがメッセー
ジを交換して通信するようなモデル.プロセス計算は,プロセスの基本動作と組
合せ方(通信)によって表現される.

\subsection{並列合成(Parallel composition)}
\[
 P | Q
\]
で,プロセス$P, Q$は並行動作していることを表す.

\subsection{通信と同期(Communication and Synchronization)}
通信路を経由.通信できる相手は,通信路(channel)がつながっている先.

\subsection{メッセージ通信・受信}
\[
 \bar{c}<x> \,\, c<x>
\]
オブジェクト指向のモデルだったりするらしい.

\subsection{同期通信路(Synchronous channel)}
メッセージが受信されるまで送信は終わらない.\footnote{Cとかの関数呼出し
は同期通信です.}

\subsection{非同期通信路(Asynchronous channel)}
送信はすぐ終わり,受信は後で可能.\footnote{ABCLとかいう言語は非同期通信
を実装した最初の言語だそうで
す.\url{http://en.wikipedia.org/wiki/Actor-Based_Concurrent_Language}}

\subsection{逐次合成(Sequential composition)}
メッセージ送受の順序関係.例えば,「受信してから送信する」など.

\subsection{簡約化規則(Reduction rules)}
(記述なし)


\section{パイ計算($\pi$ calculus)}
プロセス計算の代表的枠組み.並行版の$\lambda$計算. \footnote{萌えます
ね}

\subsection{基本要素}
\begin{description}
 \item[$P|Q$] \mbox{}\\
            並行動作

 \item[$c(x) . P$] \mbox{}\\
            $c$から$x$を受信してから$P$を実行

 \item[$\bar{c}<x> . P$] \mbox{}\\
            $c$に$x$を送信してから$P$を実行.$c?(x)$や$c!(x)$という表記
            もある.

 \item[$!P$] \mbox{}\\
            $P$の無限の複製.つまり,$P$というプロセスを無限にforkする
            (できる)ことを表す.

 \item[$0$] \mbox{}\\
            空のプロセス(nil process \footnote{nilってnihilの略から生ま
            れたんですって.関係ないけど豆知識})

 \item[$(\nu c)P$] \mbox{} \\
            通信路$c$を生成して,それを使って$P$を実行.$new c . P$とい
            う表記も.
\end{description}

\begin{myexample}{パイ計算の実例}
 \[
  (\nu z)  ( (\nu x) ( \textcolor{Green}{\bar{x}<z> . 0} | \textcolor{red}{x(y) . \bar{y} <x> . x(w) . 0}) |
 \textcolor{blue}{z(v) . \bar{v}(v) . 0})
 \]
 以下で,\textcolor{Green}{緑字を(1-1)},\textcolor{red}{赤字を
 (1-2)},\textcolor{blue}{青字を(2)}と表記する.

 この式で表される動作の概略は,次のようなものである.
 \begin{enumerate}
  \item 通信路$x$を作り,二つのプロセス(1-1),(1-2)を動かす.
        \begin{description}
         \item[(1-1)] $x$に$z$を送る
         \item[(1-2)] $x$から$y$を受信して,$y$に$x$を送り,$x$から$w$を
                    受信
        \end{description}
        
  \item (2)が通信路$z$から$v$を受信して,そこに$v$自身を送る
 \end{enumerate}

 より詳細に見てみる.
 \begin{enumerate}
  \item (1-1)から送った$z$を(1-2)は$y$に受信.
        \[
         (\nu z) ( (\nu x) (0 | \bar{z}<x> . x(w) . 0) | z(v)
        . \bar{v}<v> . 0)
        \]
  \item (1-2)が$z$に$x$を送り,(2)は$v$に受信.
        \[
         (\nu z) ( (\nu x) (0 | x (w) . 0) | \bar{x} <x> . 0)
        \]
  \item (2)が$x$に$x$を送り,(1-2)が受信
        \[
         (\nu z) ((\nu x) (0 | 0) | 0)
        \]
  \item 終了
 \end{enumerate}
\end{myexample}

\subsection{特徴}
パイ計算には,次のような重要な特徴がある.
\begin{itemize}
 \item 通信路を通信できる.
 \item それ自体でチューリング完全.つまり,パイ計算だけであらゆる計算を
       表現可能.
\end{itemize}

\begin{myexample}{真偽値}
 \begin{eqnarray*}
 &T&rue  \hspace{2ex} c(t) . c(e) . c(x) . \bar{t} <x> \\
 &F&alse  \hspace{2ex} c(t) . c(e) . c(x) . \bar{e} <x>
\end{eqnarray*}
 2つの通信路とデータ$x$を受け取って,Trueは第1の通信路に,Falseは第2のデー
 タ$x$を送る.
\end{myexample}
